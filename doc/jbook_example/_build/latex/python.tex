%% Generated by Sphinx.
\def\sphinxdocclass{jupyterBook}
\documentclass[letterpaper,10pt,english]{jupyterBook}
\ifdefined\pdfpxdimen
   \let\sphinxpxdimen\pdfpxdimen\else\newdimen\sphinxpxdimen
\fi \sphinxpxdimen=.75bp\relax
\ifdefined\pdfimageresolution
    \pdfimageresolution= \numexpr \dimexpr1in\relax/\sphinxpxdimen\relax
\fi
%% let collapsible pdf bookmarks panel have high depth per default
\PassOptionsToPackage{bookmarksdepth=5}{hyperref}
%% turn off hyperref patch of \index as sphinx.xdy xindy module takes care of
%% suitable \hyperpage mark-up, working around hyperref-xindy incompatibility
\PassOptionsToPackage{hyperindex=false}{hyperref}
%% memoir class requires extra handling
\makeatletter\@ifclassloaded{memoir}
{\ifdefined\memhyperindexfalse\memhyperindexfalse\fi}{}\makeatother

\PassOptionsToPackage{warn}{textcomp}

\catcode`^^^^00a0\active\protected\def^^^^00a0{\leavevmode\nobreak\ }
\usepackage{cmap}
\usepackage{fontspec}
\defaultfontfeatures[\rmfamily,\sffamily,\ttfamily]{}
\usepackage{amsmath,amssymb,amstext}
\usepackage{polyglossia}
\setmainlanguage{english}



\setmainfont{FreeSerif}[
  Extension      = .otf,
  UprightFont    = *,
  ItalicFont     = *Italic,
  BoldFont       = *Bold,
  BoldItalicFont = *BoldItalic
]
\setsansfont{FreeSans}[
  Extension      = .otf,
  UprightFont    = *,
  ItalicFont     = *Oblique,
  BoldFont       = *Bold,
  BoldItalicFont = *BoldOblique,
]
\setmonofont{FreeMono}[
  Extension      = .otf,
  UprightFont    = *,
  ItalicFont     = *Oblique,
  BoldFont       = *Bold,
  BoldItalicFont = *BoldOblique,
]



\usepackage[Bjarne]{fncychap}
\usepackage[,numfigreset=1,mathnumfig]{sphinx}

\fvset{fontsize=\small}
\usepackage{geometry}


% Include hyperref last.
\usepackage{hyperref}
% Fix anchor placement for figures with captions.
\usepackage{hypcap}% it must be loaded after hyperref.
% Set up styles of URL: it should be placed after hyperref.
\urlstyle{same}


\usepackage{sphinxmessages}



        % Start of preamble defined in sphinx-jupyterbook-latex %
         \usepackage[Latin,Greek]{ucharclasses}
        \usepackage{unicode-math}
        % fixing title of the toc
        \addto\captionsenglish{\renewcommand{\contentsname}{Contents}}
        \hypersetup{
            pdfencoding=auto,
            psdextra
        }
        % End of preamble defined in sphinx-jupyterbook-latex %
        

\title{DSCI 310: Historical Horse Population in Canada}
\date{Mar 19, 2022}
\release{}
\author{Tiffany Timbers \& Jordan Bourak}
\newcommand{\sphinxlogo}{\vbox{}}
\renewcommand{\releasename}{}
\makeindex
\begin{document}

\pagestyle{empty}
\sphinxmaketitle
\pagestyle{plain}
\sphinxtableofcontents
\pagestyle{normal}
\phantomsection\label{\detokenize{jbook_example::doc}}


\sphinxAtStartPar
This project explores the historical population of horses in Canada
between 1906 and 1972 for each province.

\begin{DUlineblock}{0em}
\item[] \sphinxstylestrong{\Large Data}
\end{DUlineblock}

\sphinxAtStartPar
Horse population data were sourced from the \sphinxhref{http://open.canada.ca/en/open-data}{Government of Canada’s Open Data website}.
Specifically, {[}\hyperlink{cite.jbook_example:id14}{Government of Canada, 2017}{]} and {[}\hyperlink{cite.jbook_example:id15}{Government of Canada, 2017}{]}.

\begin{DUlineblock}{0em}
\item[] \sphinxstylestrong{\large Methods}
\end{DUlineblock}

\sphinxAtStartPar
The R programming language {[}\hyperlink{cite.jbook_example:id9}{R Core Team, 2019}{]} and the following R packages were used
to perform the analysis: knitr {[}\hyperlink{cite.jbook_example:id13}{Xie, 2014}{]}, {[}\hyperlink{cite.jbook_example:id12}{Wickham, 2017}{]}, and
{[}\hyperlink{cite.jbook_example:id11}{Xie, 2016}{]}
\sphinxstyleemphasis{Note: this report is adapted from {[}\hyperlink{cite.jbook_example:id10}{Timbers, 2020}{]}.}

\begin{DUlineblock}{0em}
\item[] \sphinxstylestrong{\large Results}
\end{DUlineblock}

\begin{figure}[htbp]
\centering
\capstart

\noindent\sphinxincludegraphics{{horse_pops_plot}.png}
\caption{Horse populations for all provinces in Canada from 1906 \sphinxhyphen{} 1972}\label{\detokenize{jbook_example:horse-pop-plt}}\end{figure}

\sphinxAtStartPar
We can see from Fig. \hyperref[\detokenize{jbook_example:horse-pop-plt}]{Fig.\@ \ref{\detokenize{jbook_example:horse-pop-plt}}}
that Ontario, Saskatchewan and Alberta have had the highest horse populations in Canada.
All provinces have had a decline in horse populations since 1940.
This is likely due to the rebound of the Canadian automotive
industry after the Great Depression and the Second World War.
An interesting follow\sphinxhyphen{}up visualisation would be car sales per year for each
Province over the time period visualised above to further support this hypothesis.

\begin{sphinxuseclass}{cell}\begin{sphinxVerbatimInput}

\begin{sphinxuseclass}{cell_input}
\begin{sphinxVerbatim}[commandchars=\\\{\}]
\PYG{n}{horses\PYGZus{}sd} \PYG{o}{=} \PYG{n}{pd}\PYG{o}{.}\PYG{n}{read\PYGZus{}csv}\PYG{p}{(}\PYG{l+s+s2}{\PYGZdq{}}\PYG{l+s+s2}{../../results/horses\PYGZus{}sd.csv}\PYG{l+s+s2}{\PYGZdq{}}\PYG{p}{)}

\PYG{n}{largest\PYGZus{}sd\PYGZus{}prov} \PYG{o}{=} \PYG{n+nb}{str}\PYG{p}{(}\PYG{n}{horses\PYGZus{}sd}\PYG{p}{[}\PYG{l+s+s1}{\PYGZsq{}}\PYG{l+s+s1}{Province}\PYG{l+s+s1}{\PYGZsq{}}\PYG{p}{]}\PYG{p}{[}\PYG{l+m+mi}{0}\PYG{p}{]}\PYG{p}{)}
\PYG{n}{glue}\PYG{p}{(}\PYG{l+s+s2}{\PYGZdq{}}\PYG{l+s+s2}{largest\PYGZhy{}sd\PYGZhy{}prov}\PYG{l+s+s2}{\PYGZdq{}}\PYG{p}{,} \PYG{n}{largest\PYGZus{}sd\PYGZus{}prov}\PYG{p}{)}

\PYG{n}{horses\PYGZus{}sd\PYGZus{}noindex} \PYG{o}{=} \PYG{n}{horses\PYGZus{}sd}\PYG{o}{.}\PYG{n}{style}\PYG{o}{.}\PYG{n}{hide\PYGZus{}index}\PYG{p}{(}\PYG{p}{)}
\PYG{n}{glue}\PYG{p}{(}\PYG{l+s+s2}{\PYGZdq{}}\PYG{l+s+s2}{horses\PYGZhy{}tbl}\PYG{l+s+s2}{\PYGZdq{}}\PYG{p}{,} \PYG{n}{horses\PYGZus{}sd\PYGZus{}noindex}\PYG{p}{)}
\end{sphinxVerbatim}

\end{sphinxuseclass}\end{sphinxVerbatimInput}
\begin{sphinxVerbatimOutput}

\begin{sphinxuseclass}{cell_output}
\begin{sphinxVerbatim}[commandchars=\\\{\}]
\PYGZsq{}Saskatchewan\PYGZsq{}
\end{sphinxVerbatim}

\begin{sphinxVerbatim}[commandchars=\\\{\}]
\PYGZlt{}pandas.io.formats.style.Styler at 0x7fecb2ffd8b0\PYGZgt{}
\end{sphinxVerbatim}

\end{sphinxuseclass}\end{sphinxVerbatimOutput}

\end{sphinxuseclass}
\sphinxAtStartPar
Suppose we were interested in looking in more closely at the
province with the highest spread (in terms of standard deviation)
of horse populations. We present the standard deviations here:

\begin{figure}[htbp]
\centering
\capstart
\begin{sphinxVerbatimOutput}

\begin{sphinxuseclass}{cell_output}
\begin{sphinxVerbatim}[commandchars=\\\{\}]
\PYGZlt{}pandas.io.formats.style.Styler at 0x7fecb2ffd8b0\PYGZgt{}
\end{sphinxVerbatim}

\end{sphinxuseclass}\end{sphinxVerbatimOutput}
\caption{Standard deviation of number of horses for each province between 1940 \sphinxhyphen{} 1972}\label{\detokenize{jbook_example:horses-tbl-fig}}\end{figure}

\sphinxAtStartPar
Note that we define standard deviation (of a sample) as:
\begin{equation*}
\begin{split}s = sqrt{sum_{i = 1}^n(x_i - \bar{x})} / {n-1}.\end{split}
\end{equation*}
\sphinxAtStartPar
Additionally, note that in \hyperref[\detokenize{jbook_example:horses-tbl-fig}]{Fig.\@ \ref{\detokenize{jbook_example:horses-tbl-fig}}} we
consider the sample standard deviation of the number of horses
during the same time span as \sphinxcode{\sphinxupquote{horse\sphinxhyphen{}pops\sphinxhyphen{}plot}}.

\begin{figure}[htbp]
\centering
\capstart

\noindent\sphinxincludegraphics[width=200\sphinxpxdimen]{{horse_pop_plot_largest_sd}.png}
\caption{Horse populations for the province with the largest standard deviation}\label{\detokenize{jbook_example:id16}}\end{figure}

\sphinxAtStartPar
In \sphinxcode{\sphinxupquote{Fig. \{number\}}} we zoom in
on the province of \DUrole{pasted-inline}{\sphinxcode{\sphinxupquote{'Saskatchewan'}}}, which had the largest spread of values in
terms of standard deviation.

\sphinxAtStartPar


\begin{sphinxthebibliography}{Governme}
\bibitem[Tim20]{jbook_example:id10}
\sphinxAtStartPar
\sphinxstylestrong{missing journal in ttimbers\_horses}
\bibitem[Wic17]{jbook_example:id12}
\sphinxAtStartPar
\sphinxstylestrong{missing publisher in tidyverse}
\bibitem[Xie14]{jbook_example:id13}
\sphinxAtStartPar
Yihui Xie. \sphinxstyleemphasis{knitr: A Comprehensive Tool for Reproducible Research in R}. Chapman and Hall/CRC, 2014. ISBN 978\sphinxhyphen{}1466561595. URL: \sphinxurl{http://www.crcpress.com/product/isbn/9781466561595}.
\bibitem[Xie16]{jbook_example:id11}
\sphinxAtStartPar
Yihui Xie. \sphinxstyleemphasis{bookdown: Authoring Books and Technical Documents with R Markdown}. Chapman and Hall/CRC, Boca Raton, Florida, 2016. ISBN 978\sphinxhyphen{}1138700109. URL: \sphinxurl{https://bookdown.org/yihui/bookdown}.
\bibitem[GovernmentoCanada17a]{jbook_example:id14}
\sphinxAtStartPar
\sphinxstylestrong{missing journal in horses1}
\bibitem[GovernmentoCanada17b]{jbook_example:id15}
\sphinxAtStartPar
\sphinxstylestrong{missing journal in horses2}
\bibitem[RCTeam19]{jbook_example:id9}
\sphinxAtStartPar
\sphinxstylestrong{missing publisher in R}
\end{sphinxthebibliography}







\renewcommand{\indexname}{Index}
\printindex
\end{document}